\documentclass{article}


\usepackage{authblk}
\usepackage{listings, chngcntr}
\usepackage{multicol}
\usepackage{textcomp}
\usepackage{float}
\usepackage[T1]{fontenc}
\usepackage{indentfirst}
\usepackage{graphicx}
\usepackage{array}
\usepackage{caption} 
\usepackage{hyperref}
\usepackage{verbatim}
\usepackage{float}
\usepackage{subcaption}
\usepackage{gensymb}
\usepackage{amsmath}
\usepackage{geometry}
\usepackage{multirow}
\usepackage{apacite}
\usepackage{listings}


\geometry{
    a4paper,
    left=30mm,
    right=30mm,
    top=30mm,
    bottom=40mm
}

\begin{document}

\title{Noise Floor Estimation and The Performance Check of the Operational Amplifier}
\author[1]{Woojin Han}
\affil[1]{Seoul National University, Seoul 151-747, Korea}
\maketitle

\begin{abstract}
    abcd
\end{abstract}

\section{Introduction}
The original purpose of this report and the experiment was to find electron spin resonance(ESR) in well-prepared settings, but the goal is not achievable due to insufficient lab settings.
Therefore in this report, I elaborate on the detailed mechanisms of the ESR of the Nitrogen-Vacancy Center(NV center) in mathematical models and the apparatus we have in the lab.
Moreover, the limitation we have shall be revealed by the appropriate noise estimation so that the report may help the settings work well.
(introduction of the sections)

\subsection{Theory: NV Center}
The Nitrogen-Vacancy center (NV center) is the crystal structure that has a nitrogen atom that is adjacent to a vacancy in the diamond lattice.
The negatively
\cite{nv} % general explanation of nv center
\cite{nvphonon} % nv phonon interaction
\subsection{ESR}

\subsection{Lavi rotation}

\section{Experiment}


\bibliography{esr_ref}
\bibliographystyle{apacite}
\end{document}
